\chapter{Grammar}

\section{Word order}

Ségēna has nominative-accusative alignment, but no strict word order. Usually
the speaker will mention emphasized content first and secondary content last.
However, if the speaker doesn't want to stress any specific parts of the
sentence, the order considered most neutral is usually subject-verb-object or
verb-subject-object.

For example, to the listener the object usually makes sense only once the verb
is known, so the object would naturally come after the verb. If the verb doesn't
make sense when the subject is not known, then the subject would be placed
before the verb, thus ending up with subject-verb-object.

\section{Nouns}

Modern Ségēna nouns decline in two cases, nominative and accusative. Many nouns
have identical nominative and accusative forms. Those that don't, can be
recognized by a final ‘s’ in their nominative form. If the noun is regular, its
accusative form is equivalent to its nominative form without the trailing ‘s’.

\section{Verbs}

Ségēna has two main classes of verbs: durative and punctual. Depending on the
verb's class, it can be conjugated in four or five aspects (see below).
Additionally, all verbs have present, past, and future forms.

\subsection{Durative verbs}

Durative verbs represent actions that have a starting point, a duration, and an
ending point; for example "eat" and "think". They are conjugated in four
aspects:

\begin{itemize}
	\item continuous
	\item habitual
	\item passive continuous
	\item infinitive
\end{itemize}

Example conjugation of a durative verb \emph{enda-}, "eat":

\begin{table}[H]
\begin{tabular}{ l|l|l|l }
	\emph{enda-}              & Present (\emph{-a-})          & Past (\emph{-u-})              & Future (\emph{-e-y-})                  \\
	\hline
	Continuous (\emph{-a})    & \emph{enda}, is eating        & \emph{endua}, was eating       & \emph{endeya}, will be eating          \\
	Habitual (\emph{-s-a})    & \emph{endasa}, eats (habit)   & \emph{endusa}, used to eat     & \emph{endesha}, will be eating (habit) \\
	Pass. cont. (\emph{-l-a}) & \emph{endala}, is being eaten & \emph{endula}, was being eaten & \emph{endelya}, will be being eaten    \\
	Infinitive (\emph{-i})    & \emph{endai}, to eat          & \emph{endui}, to have eaten    & \emph{endeyi}, to be about to eat      \\
\end{tabular}
\end{table}

\subsection{Punctual verbs}

Punctual verbs describe actions that are instantaneous or very short, and thus
don't have a clear duration; for example "hit" and "die". They are conjugated in
five aspects:

\begin{itemize}
	\item perfective
	\item iterative
	\item habitual
	\item passive continuous
	\item infinitive
\end{itemize}

There's no present perfective form.

Example conjugation of a punctual verb \emph{kata-}, "hit":

\begin{table}[H]
\begin{tabular}{ l|l|l|l }
	\emph{kata-}              & Present (\emph{-a-})          & Past (\emph{-u-})              & Future (\emph{-y-})                     \\
	\hline
	Perfective (\emph{-a})    & —                             & \emph{katua}, hit              & \emph{katya}, will hit                  \\
	Iterative (\emph{-t-a})   & \emph{katata}, is hitting     & \emph{katatua}, was hitting    & \emph{katatya}, will be hitting         \\
	Habitual (\emph{-s-a})    & \emph{katasa}, hits (habit)   & \emph{katusa}, used to hit     & \emph{katasha}, will be hitting (habit) \\
	Pass. cont. (\emph{-l-a}) & \emph{katala}, is being hit   & \emph{katula}, was being hit   & \emph{katelya}, will be being hit       \\
	Infinitive (\emph{-i})    & \emph{katai}, to hit          & \emph{katui}, to have hit      & \emph{katyai}, to be about to hit       \\
\end{tabular}
\end{table}

\section{Adjectives}

Adjectives can be used as verbs.

\section{Pronouns}

\subsection{Personal pronouns}

Because Ségēna doesn't distinguish between singular and plural most of the time,
personal pronouns only have separate singular/plural forms in the 3rd person,
which is a remnant from times when a distinct plural form was used.

\begin{table}[H]
\begin{tabularx}{\textwidth}{ Y|Y|Y|Y }
	\multirow{2}{*}{Includes the speaker(s)} & \multirow{2}{*}{Includes the addressee(s)} & \multicolumn{2}{c}{Includes third party/parties} \\
	                                         &                                            & Singular                & Plural                 \\
	\hline
	\emph{ya} – I, we (exclusive)            & \emph{za} – you (singular or plural)      & \emph{ea} – (last-mentioned); \emph{īs} – (earlier-mentioned) & \emph{sha} – they \\
	\multicolumn{2}{c|}{\emph{na} – we (inclusive)}                                       & —                       & —                      \\
\end{tabularx}
\end{table}

\subsection{Demonstrative pronouns}

Demonstrative pronouns in Ségēna come in three varieties:

\begin{itemize}
	\item proximal (1st person, near the speaker)
	\item medial (2nd person, near the addressee)
	\item distal (3rd person, far from both)
\end{itemize}

\begin{table}[H]
\begin{tabular}{ c|c|c|c }
	         & Proximal   & Medial     & Distal      \\
	\hline
	Singular & \emph{ea}? & \emph{tos} & \emph{īs}?  \\
	Plural   & \emph{za}? & ?          & \emph{sha}? \\
\end{tabular}
\end{table}

\section{Prepositions}

Ségēna makes extensive use of prepositions.

\begin{table}[H]
\begin{tabular}{ c|c|c }
	Preposition & Meaning                    & Example                                                                       \\
	\hline
	\emph{a}    & marks the object of a verb & \emph{endai a mar} "to eat flesh" (\emph{endai} "to eat", \emph{mar} "flesh") \\
	…           & …                          & …                                                                             \\
\end{tabular}
\end{table}

\section{Numerals}

Ségēna uses an octal numeral system for counting to 16, and then switches to a
hexadecimal system for larger numbers:

\begin{table}[H]
\begin{tabular}{ c|c|c }
	        & Cardinal                   & Ordinal             \\
	\hline
	0       & \emph{uo}                  &                     \\
	1       & \emph{ēa}                  & \emph{eadis}        \\
	2       & \emph{hya}                 & \emph{hyadis}       \\
	3       & \emph{yáva}                & \emph{yaodis}       \\
	4       & \emph{yéla}                & \emph{yéldis}       \\
	5       & \emph{sféra}               & \emph{sférdis}      \\
	6       & \emph{ista}                & \emph{isdis}        \\
	7       & \emph{khár}                & \emph{khárdis}      \\
	8       & \emph{sadan}               & \emph{sadandis}     \\
	9       & \emph{ēadan}               & \emph{ēadandis}     \\
	10      & \emph{hyadan}              & \emph{hyadandis}    \\
	11      & \emph{yaodan}              & \emph{yaodandis}    \\
	12      & \emph{yeldan}              & \emph{yeldandis}    \\
	13      & \emph{sferdan}             & \emph{sferdandis}   \\
	14      & \emph{istadan}             & \emph{istadandis}   \\
	15      & \emph{khardan}             & \emph{khardandis}   \\
	16      & \emph{alda}                & \emph{aldis}        \\
	17      & \emph{alda-i-ēa}           & \emph{alda-i-eadis} \\
	…       & …                          & …                   \\
	32      & \emph{hyalda}              & \emph{hyaldis}      \\
	48      & \emph{yavalda}             & \emph{yavaldis}     \\
	64      & \emph{yelalda}             & \emph{yelaldis}     \\
	80      & \emph{sferalda}            & \emph{sferaldis}    \\
	96      & \emph{istalda}             & \emph{istaldis}     \\
	112     & \emph{kharalda}            & \emph{kharaldis}    \\
	128     & \emph{sadanalda}           & \emph{sadanaldis}   \\
	…       & …                          & …                   \\
	256     & \emph{gala}                & \emph{galadis}      \\
	…       & …                          & …                   \\
\end{tabular}
\end{table}
