\chapter{Phonology}

The sounds of Ségēna are similar to those of English, with the following
exceptions:

\begin{itemize}
	\item voiceless palatal fricative [ç]
	\item voiced alveolar trill [r]
	\item close front rounded vowel [y]
\end{itemize}

\section{Consonants}

\addvbuffer[15pt]{\begin{tabularx}{\textwidth}{ c|Y|Y|Y|Y|Y|Y|Y|Y|Y }
	              & Bilabial & Labio-dental & Dental & Alveolar & Palato-alveolar & Palatal & Labio-velar & Velar & Glottal \\
	\hline
	Nasal         & m        &              &        & n        &                 &         &             & ŋ     &         \\
	Stop          & p · b    &              &        & t · d    &                 &         &             & k · g &         \\
	Fricative     &          & f · v        & ð      & s · z    & \sh{} · \zh{}   & ç       &             &       & h       \\
	Affricate     &          &              &        &          & t\sh{} · d\zh{} &         &             &       &         \\
	Approximant   &          &              &        &          &                 & j       & w           &       &         \\
	Trill         &          &              &        & r        &                 &         &             &       &         \\
	Lateral appr. &          &              &        & l        &                 &         &             &       &         \\
\end{tabularx}}

The Latin orthography of Ségēna mostly follows the phonetic alphabet, except for
the phonemes in the table below:

\addvbuffer[15pt]{\begin{tabular}{ c|c }
	Phoneme & Grapheme      \\
	\hline
	ð       & \grapheme{dh} \\
	\sh{}   & \grapheme{sh} \\
	\zh{}   & \grapheme{zh} \\
	t\sh{}  & \grapheme{ch} \\
	d\zh{}  & \grapheme{j}  \\
	ç       & \grapheme{hy} \\
	j       & \grapheme{y}  \\
	ŋ       & \grapheme{ng} \\
\end{tabular}}

Ségēna also has the aspirated consonants \textipa{/p\super{h} t\super{h}
k\super{h}/} which are written \grapheme{ph th kh} respectively, as well as the
palatalized alveolar lateral approximant \textipa{/l\super{j}/}, written
\grapheme{ly} (which in an inter-syllabic position represents the consonant
cluster cluster \textipa{/lj/}.

\section{Vowels}

\addvbuffer[15pt]{\begin{tabular}{ c|c|c|c|c }
	                & \multicolumn{2}{c|}{Front} & \multirow{2}{*}{Central, unrounded} & \multirow{2}{*}{Back, rounded} \\
	                & Unrounded    & Rounded     &                                     &                                \\
	\hline
	Close           & i \grapheme{i} & y \grapheme{ÿ} &                                & u \grapheme{u}                 \\
	Close-mid / mid & e/\textipa{\|`e} \grapheme{e} & &                                & o/\textipa{\|`o} \grapheme{o}  \\
	Open            &              &             & ä \grapheme{a}                      &                                \\
\end{tabular}}

Long vowels are denoted with a macron: \grapheme{ā ē ī ō ū ȳ}. Medium-length
vowels are denoted with an acute accent: \grapheme{á é í ó ú ý}. Note that
marking the diaeresis of \grapheme{ÿ} is redundant when marking vowel length
because the consonant \grapheme{y} cannot receive a macron or acute accent.

\subsection{Diphthongs}

Ségēna has the following diphthongs:

\addvbuffer[15pt]{\begin{tabular}{ c|c|c|c|c|c }
	   & a- & e- & i- & o- & u- \\
	\hline
	-a & —  & ea & ia & oa & ua \\
	-e & ae & —  &    & oe &    \\
	-i & ai & ei & —  &    &    \\
	-o & ao & eo &    & —  &    \\
	-u & au & eu &    & ou & —  \\
\end{tabular}}

\section{Phonotactics}

No consonant clusters are allowed at word-initial positions.
